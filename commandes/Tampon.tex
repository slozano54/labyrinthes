% Une commande pour un tampon d'indication du niveau
% #1 --> x,y
% #2 --> couleur cadre
% #3 --> couleur texte
% #4 --> texte

\newcommand{\tampon}[4]{
  % l'origine est mise en bas à gauche, l'unité est le centimètre
  \begin{tikzpicture}[remember picture, overlay,shift={(current page.south west)}] 
    \node[shift={(#1)},draw=#2,rounded corners,rectangle,inner sep=5mm,rotate=30,very thick,opacity=.9]
    {
      \begin{tabular}{c}
        \textcolor{#3}{#4}
      \end{tabular}
    };
  \end{tikzpicture}
}

% Nouvelle commande qui remplacera la précédente
% Les appels sont à modifier
% Une commande pour faire un tampon avec un système de clefs/valeurs
% On définit le trousseau par défaut
\setKVdefault[Tampon]{%
  Angle=30,%
  Abscisse=2,%
  Ordonnee=27,%
  CouleurCadre=red,%
  CouleurTexte=red,%
  Largeur=10cm,
  TexteAModifier=false
}
% On définit une nouvelle clef au besoin
% Cela signifie que si la clef Texte est utilisée, la booléenne TexteAModifier passe à true
\defKV[Tampon]{Texte=\setKV[Tampon]{TexteAModifier}}

\NewDocumentCommand\Tampon{o}{%
  \useKVdefault[Tampon]% On revient aux valeurs par défaut, équivalent à restoreKV[Tampon]
  \setKV[Tampon]{#1}% On lit les arguments optionnels
  \xdef\TamponAngle{\useKV[Tampon]{Angle}}
  \xdef\TamponAbscisse{\useKV[Tampon]{Abscisse}}
  \xdef\TamponOrdonnee{\useKV[Tampon]{Ordonnee}}
  \xdef\TamponCouleurCadre{\useKV[Tampon]{CouleurCadre}}
  \xdef\TamponCouleurTexte{\useKV[Tampon]{CouleurTexte}}
  \def\TamponTexte{
    \ifboolKV[Tampon]{TexteAModifier}
      {\useKV[Tampon]{Texte}}
      {Amazing tampon !}
  }
  \begin{tikzpicture}[remember picture, overlay,shift={(current page.south west)}] 
    \node[shift={(\TamponAbscisse,\TamponOrdonnee)},draw=\TamponCouleurCadre,rounded corners,rectangle,inner sep=5mm,rotate=\TamponAngle,very thick,opacity=.9]
    {%
      \begin{minipage}{\useKV[Tampon]{Largeur}}
        \begin{center}
          \vspace*{-2mm}
          \color{\TamponCouleurTexte}
          \TamponTexte\\\vspace*{-5mm}
        \end{center}
      \end{minipage}
    };
  \end{tikzpicture}
}