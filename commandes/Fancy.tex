%==========================================
% Entetes et pieds de pages
%==========================================
%   LE : champ gauche pour les pages paires
%   LO : champ gauche pour les pages impaires
%   C : champ central pour toutes les pages
%   CE : champ central pour les pages paires
%   CO : champ central pour les pages impaires
%   R : champ droit pour toutes les pages
%   RE : champ droit pour les pages paires
%   RO : champ droit pour les pages impaires

\fancypagestyle{front}{%
                \fancyhf{}%on vide l'en-tête
                \fancyfoot[C]{page \thepage}%
                \renewcommand{\headrulewidth}{0pt}%trait horizontal pour l'en-tête
                \renewcommand{\footrulewidth}{0.4pt}%trait horizontal pour le pied de page
}

% On redéfinit le style plain pour corriger le style imposé par la classe report aux \chapter
\fancypagestyle{plain}{%
\renewcommand{\headrulewidth}{0pt}
\renewcommand{\footrulewidth}{1pt}
\fancyfoot{}   % réinitialiser les pieds
\fancyhead{}   % réinitialiser les entêtes
\fancyfoot[C]{%
\\\textbf{\thepage/\pageref{LastPage}}} 
\fancyfoot[L]{\currentSectionTitle \par Sébastien LOZANO - \href{https://mathslozano.fr/}{\textbf{https://mathslozano.fr/}}}
\fancyfoot[R]{\href{https://irem.univ-lorraine.fr/}{\textbf{https://irem.univ-lorraine.fr/}}\\\href{https://creativecommons.org/licenses/by-nc-sa/4.0/deed.fr}{CC BY-NC-SA}}
}