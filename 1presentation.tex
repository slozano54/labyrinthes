\section{La nuit du jeu 2023 fait germer une idée}
Lors de \href{https://partage.apmeplorraine.fr/petitVert/PV155.pdf#page=5}{la nuit du jeu mathématique de juin 2023} à Frouard,
André STEFF animait un atelier autour du labyrinthe. Isabelle MILLERAND séduite par l'activité s'est imaginée la mise en place d'un projet autour 
de ce thème en 6\up{e}, niveau dans lequelle elle aborde les textes mythologiques. Elle propose alors à Sébastien LOZANO de s'occuper de l'aspect mathématique 
de la chose.

\dots

\section{Objectifs et liens avec les programmes demathématiques}
\subsection{Notions abordées}
    \begin{itemize}
        \item[$\leadsto$] Dénombrement.
        \item[$\leadsto$] Repérage.
        \item[$\leadsto$] Géométrie élémentaire.
    \end{itemize}

\subsection{Compétences développées}
\begin{center}
  \renewcommand{\arraystretch}{1.5}
{\footnotesize
\begin{tabular}{|m{0.25\linewidth}|m{0.67\linewidth}|}
    \hline
    {\textbf{Chercher}\par Domaines du socle : 2, 4}
    & {Tester, essayer, valider, corriger une démarche. (C2)\par
    Extraire des informations, les organiser, les confronter à ses connaissances. (C3)}\\
    \hline
    {\textbf{Représenter}\par Domaines du socle : 1, 5}
    & {Utiliser les représentations des nombres. (C2)\par 
    Produire et utiliser les représentations des nombres. (C2)\par 
    Passer d’un mode de représentation à un autre. (C4)
    }\\
    \hline
    {\textbf{Raisonner}\par Domaines du socle : 2, 3, 4}
    & {Raisonner collectivement. (C2)\par 
    Justifier, argumenter. (C2)
    }\\
    \hline
    {\textbf{Calculer}\par Domaines du socle : 4}
    & {Calculer avec des nombres. (C2)\par 
    Contrôler les calculs. (C2)\par 
    Calculer avec des lettres, des algorithmes\dots{} (C4)
    }\\
    \hline
    {\textbf{Communiquer}\par Domaines du socle : 1, 3}
    & {Communiquer pour expliquer, argumenter et comprendre autrui. (C3)\par
    Communiquer pour porter un regard critique. (C4)
    }\\
    \hline
\end{tabular}
}
\renewcommand{\arraystretch}{1}
\end{center}

