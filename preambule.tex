%==========================================
% Compatibilité luaLaTeX
%==========================================
% 
% Paquet profCollege
\usepackage{ProfCollege}

%==========================================
% Modifictaion pour la compilation en mode luaLaTeX
%==========================================
% Font
\usepackage[warnings-off={mathtools-colon,mathtools-overbracket}]{unicode-math}
\setmainfont{TeX Gyre Heros}
\setmathfont{TeX Gyre Schola Math}

% Pour un symbole necessaire à l'environnement Tableur du paquet ProfCollege
\setmathfont{STIX Two Math}[
  range={\blacktriangledown}
]
%==========================================

% Pour gérer la géométrie de la page.
\usepackage[top=1cm,bottom=2cm,left=1.5cm,right=1.5cm,nohead]{geometry}
% Pour les espaces
\usepackage{setspace}
% Pour une courte table des matières dans la partie des liminaires.
\usepackage{shorttoc}
% Pour utiliser les usages français grâce au <french> de l'option de classe.
\usepackage{babel}
% Pour les liens
% \usepackage{hyperref}
\usepackage[luatex]{hyperref}
\hypersetup{
    colorlinks=true,% On active la couleur pour les liens. Couleur par défaut rouge
    linkcolor=purple,% On définit la couleur pour les liens internes
    % filecolor=magenta,% On définit la couleur pour les liens vers les fichiers locaux      
    % urlcolor=blue,% On définit la couleur pour les liens vers des sites web
    pdfauthor={Sébastien LOZANO}
    pdftitle={Labyrinthes},% On définit un titre pour le document pdf
    % pdfpagemode=FullScreen,% On fixe l'affichage par défaut à plein écran
    }

%==========================================
% Style des titres avec le paquet titlesec
%==========================================
\usepackage[explicit]{titlesec}
%==========================================
% Style des titres avec le paquet titlesec
%==========================================

% https://borntocode.fr/latex-personnaliser-les-titres-chapter/
% https://mirror.ibcp.fr/pub/CTAN/macros/latex/contrib/titlesec/titlesec.pdf 
% https://tex.stackexchange.com/questions/30432/styling-the-part-page
% https://www.youtube.com/watch?v=p2IPlxJd-5E

\titleclass{\part}{top} % make part like a chapter
\titleformat
    {\part}
    [display]
    {\centering\normalfont\large\scshape\bfseries}
    {\rule[3pt]{0.15\linewidth}{3pt}\quad {\color{red} \partname}\quad \rule[3pt]{0.15\linewidth}{3pt}}
    {0pt}
    {\rule{\linewidth}{0.5pt}\break\Huge}
    [\vspace{-0.5\baselineskip}{\color{red}#1}\break \rule{\linewidth}{0.5pt}]
\titlespacing*{\part}{0pt}{0pt}{3\baselineskip}

\titleclass{\chapter}{straight} % make chapter like a section (no newpage)
\titleformat
    {\chapter}
    [block]
    {\LARGE\scshape\bfseries}
    {\thechapter\hspace{\baselineskip}{\color{red} #1}}
    {\baselineskip}
    {}
    [\hrule\vspace{2pt}\hrule]
\titlespacing*{\chapter}{0pt}{10pt}{\baselineskip}    

\titleformat
    {\section}
    [block]
    {\Large\bfseries}
    {\thesection\hspace{\baselineskip}{\color{red} #1}}
    {\baselineskip}
    {}
    []
\titlespacing*{\section}{0pt}{10pt}{\baselineskip} 

\titleformat
    {\subsection}
    [block]
    {\large\bfseries}
    {{\color{red} #1}}
    {\baselineskip}
    {}
    []
\titlespacing*{\subsection}{0pt}{10pt}{\baselineskip}

% Réinitiliser les numérotations de chapitre à chaque changement de partie
% https://www.mathematex.fr/viewtopic.php?t=17461
\makeatletter
\@addtoreset{chapter}{part}
\makeatother


%==========================================
% Entetes et pieds de pages
%==========================================
\usepackage{lastpage}
\usepackage{fancyhdr}
%==========================================
% Entetes et pieds de pages
%==========================================
%   LE : champ gauche pour les pages paires
%   LO : champ gauche pour les pages impaires
%   C : champ central pour toutes les pages
%   CE : champ central pour les pages paires
%   CO : champ central pour les pages impaires
%   R : champ droit pour toutes les pages
%   RE : champ droit pour les pages paires
%   RO : champ droit pour les pages impaires

\fancypagestyle{front}{%
                \fancyhf{}%on vide l'en-tête
                \fancyfoot[C]{page \thepage}%
                \renewcommand{\headrulewidth}{0pt}%trait horizontal pour l'en-tête
                \renewcommand{\footrulewidth}{0.4pt}%trait horizontal pour le pied de page
}

% On redéfinit le style plain pour corriger le style imposé par la classe report aux \chapter
\fancypagestyle{plain}{%
\renewcommand{\headrulewidth}{0pt}
\renewcommand{\footrulewidth}{1pt}
\fancyfoot{}   % réinitialiser les pieds
\fancyhead{}   % réinitialiser les entêtes
\fancyfoot[C]{%
\\\textbf{\thepage/\pageref{LastPage}}} 
\fancyfoot[L]{\currentSectionTitle \par Sébastien LOZANO - \href{https://mathslozano.fr/}{\textbf{https://mathslozano.fr/}}}
\fancyfoot[R]{\href{https://irem.univ-lorraine.fr/}{\textbf{https://irem.univ-lorraine.fr/}}\\\href{https://creativecommons.org/licenses/by-nc-sa/4.0/deed.fr}{CC BY-NC-SA}}
}

%==========================================
% Quelques constantes et commandes
%==========================================
\newcommand{\myHeader}{%
    Labyrinthes \hspace*{\stretch{1}} Construction\\%  
    Algorithme \hspace*{\stretch{1}} Paquet \LaTeX{} ProfCollege%
}
\newcommand{\myHeaderTitle}{%
  {\Huge PROJET PLURIDISCIPLINAIRE}\\%
  \medskip%
  {\LARGE MATHS / FRANÇAIS}%
}
\newcommand{\myTitle}{LABYRINTHES}
\newcommand{\myAuthor}{%
    Auteurs\\
    {\setmainfont{QTCaslanOpen}\large% 
    \begin{multicols}{3}       
        Isabelle MILLERAND\\
        André STEFF\\
        Sébastien LOZANO\\
    \end{multicols}
    }

    \smallskip
    {\normalsize 	isabelle.millerand@ac-nancy-metz.fr \& andre.stef@univ-lorraine.fr \& sebastien.lozano@ac-nancy-metz.fr}
}
\newcommand{\myDate}{Mis à jour le \today}
\newcommand{\myThanks}{
  \begin{itemize}
    \item[$\leadsto$] Christophe POULAIN pour son paquet ProfCollege, ses échanges constructifs et son aide précieuse en \LaTeX{};
    \item[$\leadsto$] \dots
    \item[$\leadsto$] \dots
  \end{itemize}
}

\newcommand{\currentSectionTitle}{Labyrinthes}

\makeatletter
\def\myThickRuleFill{\leavevmode \leaders \hrule height 1ex \hfill \kern \z@}
\makeatother

%==========================================
% Tampon
%==========================================
% Une commande pour un tampon d'indication du niveau
% #1 --> x,y
% #2 --> couleur cadre
% #3 --> couleur texte
% #4 --> texte

\newcommand{\tampon}[4]{
  % l'origine est mise en bas à gauche, l'unité est le centimètre
  \begin{tikzpicture}[remember picture, overlay,shift={(current page.south west)}] 
    \node[shift={(#1)},draw=#2,rounded corners,rectangle,inner sep=5mm,rotate=30,very thick,opacity=.9]
    {
      \begin{tabular}{c}
        \textcolor{#3}{#4}
      \end{tabular}
    };
  \end{tikzpicture}
}

% Nouvelle commande qui remplacera la précédente
% Les appels sont à modifier
% Une commande pour faire un tampon avec un système de clefs/valeurs
% On définit le trousseau par défaut
\setKVdefault[Tampon]{%
  Angle=30,%
  Abscisse=2,%
  Ordonnee=27,%
  CouleurCadre=red,%
  CouleurTexte=red,%
  Largeur=10cm,
  TexteAModifier=false
}
% On définit une nouvelle clef au besoin
% Cela signifie que si la clef Texte est utilisée, la booléenne TexteAModifier passe à true
\defKV[Tampon]{Texte=\setKV[Tampon]{TexteAModifier}}

\NewDocumentCommand\Tampon{o}{%
  \useKVdefault[Tampon]% On revient aux valeurs par défaut, équivalent à restoreKV[Tampon]
  \setKV[Tampon]{#1}% On lit les arguments optionnels
  \xdef\TamponAngle{\useKV[Tampon]{Angle}}
  \xdef\TamponAbscisse{\useKV[Tampon]{Abscisse}}
  \xdef\TamponOrdonnee{\useKV[Tampon]{Ordonnee}}
  \xdef\TamponCouleurCadre{\useKV[Tampon]{CouleurCadre}}
  \xdef\TamponCouleurTexte{\useKV[Tampon]{CouleurTexte}}
  \def\TamponTexte{
    \ifboolKV[Tampon]{TexteAModifier}
      {\useKV[Tampon]{Texte}}
      {Amazing tampon !}
  }
  \begin{tikzpicture}[remember picture, overlay,shift={(current page.south west)}] 
    \node[shift={(\TamponAbscisse,\TamponOrdonnee)},draw=\TamponCouleurCadre,rounded corners,rectangle,inner sep=5mm,rotate=\TamponAngle,very thick,opacity=.9]
    {%
      \begin{minipage}{\useKV[Tampon]{Largeur}}
        \begin{center}
          \vspace*{-2mm}
          \color{\TamponCouleurTexte}
          \TamponTexte\\\vspace*{-5mm}
        \end{center}
      \end{minipage}
    };
  \end{tikzpicture}
}

%==========================================
% Commandes Puissance Quatre
%==========================================
% Commandes spécifiques labyrinthe

%==========================================
% Reglages table des matières
%==========================================
\setcounter{tocdepth}{1}