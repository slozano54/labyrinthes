
\section{Description des temps du projet}
\subsection{Parlons labyrinthe en français}
Durée ? Description du contenu ?

\dots

\subsection{Parlons labyrinthe en mathématiques}
2 séances d'une heure.

Qu'est qu'un labyrinthe ?

Maths et labytinthe.

Parcourir un labyrinthe.

Algorithme de sortie.

À la fin de la deuxième séance, on précise que c'est leur labyrinthe que l'on va construire.

Il y aura des règles de construction à respecter.

\subsection{Proposition de labyrinthes}
Les élèves auront à concevoir des labyrinthes.

L'activité se fera sous forme de DM navette classe/maison. Non noté !

À l'issue, un labyrinthe sera choisi.
\subsection{Réalisation du labyrinthe au labo de maths}
Une première réalisation se fera dans le laboratoire de maths.
Matériel necessaire ?

Des questions :
\begin{itemize}
    \item Comment réaliser ce labyrinthe ? 
    \begin{itemize}
        \item Ordre de construction ?
        \item Comment faire des perpendiculaires ?
        \item Autres problèmes a priori ?
    \end{itemize}
\end{itemize}

Craie ? Peinture ? Adhesif ? 

\subsection{Réalisation du labyrinthe dans la cour}
Craie ? Peinture ? Adhesif ? 

\section{Évaluation de l'activité, intérêts pédagogiques}
\begin{itemize}
    \item[$\leadsto$] \dots
    \item[$\leadsto$] \dots
    \item[$\leadsto$] \dots    
    \item[$\leadsto$] \dots
\end{itemize}