% Durée ? Description du contenu ?

\textbf{À la Découverte du Labyrinthe : Entre Mythes, Ruses et Apprentissages}

La figure du labyrinthe fascine depuis des siècles, traversant différentes civilisations et générations. Cette séquence pédagogique propose une exploration approfondie de cette énigmatique structure, abordant des objets d'étude tels que le monstre aux limites de l'humain et les stratégies de résistance face à l'adversité. Les supports variés, du film à la littérature, guideront les élèves dans une quête intellectuelle et imaginative.

Le monstre aux limites de l'humain et la résistance au plus fort sont les axes de cette séquence. À travers le film "Icare" de Carlo Vogele, le livre "Ariane contre le Minotaure" de Marie-Odile Hartmann, et la bande dessinée "Thésée" de Yvan Pommaux, les élèves plongeront dans les méandres de ces thèmes universels.

La séquence vise à découvrir les caractéristiques du labyrinthe en tant que figure présente dans de nombreuses civilisations, lieu d'égarement et endroit dangereux. Plus profondément, les élèves exploreront le labyrinthe comme une quête de soi-même, la victoire de l'intelligence humaine, et un chemin vers l'au-delà.

Les \oe uvres sélectionnées ouvriront les portes de la compréhension du thème du labyrinthe. Du film à la BD, les élèves seront immergés dans des univers diversifiés, encourageant une approche pluridisciplinaire.

Une découverte virtuelle des labyrinthes sur le site de la BNF, la résolution de labyrinthes sur support papier, et la réflexion sur les qualités nécessaires pour sortir d'un labyrinthe poseront les bases de la séquence.

L'exploration des sens propre et figuré du vocabulaire, l'utilisation des connecteurs spatiaux, l'identification des temps de l'indicatif comme des labyrinthes temporels, et la maîtrise des chaînes d'accord dans le groupe nominal enrichiront la compréhension linguistique des élèves.

Des activités pratiques telles que guider un camarade les yeux bandés, exprimer des émotions et la lecture oralisée renforceront les compétences orales des élèves.

Les élèves seront amenés à décrire un labyrinthe, à narrer les ruses employées pour s'échapper, et à exprimer les sensations vécues à l'intérieur d'un labyrinthe, stimulant ainsi leur créativité et leur maîtrise de l'écriture.

La réalisation d'affiches individuelles synthétisant les apprentissages en mathématiques et en français permettra aux élèves de présenter leurs découvertes à la communauté éducative. Un labyrinthe grand nature sera également réalisé au sol dans la cour de récréation.

La visite d'un labyrinthe de paille et un jeu de piste au Musée des Beaux-Arts offriront des expériences concrètes liées au thème du labyrinthe.

Cette séquence s'inscrit dans une approche transdisciplinaire, impliquant les mathématiques, les arts plastiques et l'éducation physique et sportive, pour une compréhension holistique du labyrinthe.

En conclusion, cette séquence offre une plongée captivante dans la complexité du labyrinthe, alliant réflexion intellectuelle, exploration artistique et expériences concrètes, tout en favorisant le développement de compétences linguistiques et communicationnelles.

\textbf{Séance}

De mon côté je vais rajouter du vocabulaire avec la formation des mots ( influence grecque et latine) et celui du champ lexical de la peur et de l’enfermement . Le Minotaure se faisant harceler, je me demande si je ne vais pas aussi tirer ce fil … 